% !TEX TS-program = XeLaTeX
% !TEX encoding = UTF-8 Unicode

%%%%%%%%%%%%%%%%%%%%%%%%%%%%%%%%%%%%%%%%%%%%%%%%%%%%%%%%%%%%%%%%%%%%%%
%
%	大连理工大学硕士论文 XeLaTeX 模版 —— 格式文件 format.tex
%	版本:0.71
%	最后更新:2010.12.22
%	修改者:Yuri (E-mail: yuri_1985@163.com)
%	编译环境:Ubuntu 10.04 + TeXLive 2010 + TeXworks
%             Windows XP SP3 + CTeXLive 2009 + WinEdt 5.6
%
%%%%%%%%%%%%%%%%%%%%%%%%%%%%%%%%%%%%%%%%%%%%%%%%%%%%%%%%%%%%%%%%%%%%%%

%%%%%%%%%%%%%%%%%%%%%%%%%%%%%%%%%%%%%%%%%%%%%%%%%%%%%%%%%%%%%%%%%%%%%%
% 页面设置
%%%%%%%%%%%%%%%%%%%%%%%%%%%%%%%%%%%%%%%%%%%%%%%%%%%%%%%%%%%%%%%%%%%%%%
% A4 纸张
\setlength{\paperwidth}{21.0cm}
\setlength{\paperheight}{29.7cm}
% 设置正文尺寸大小
\setlength{\textwidth}{16.1cm}
\setlength{\textheight}{22.2cm}
% 设置正文区在正中间
\newlength \mymargin
\setlength{\mymargin}{(\paperwidth-\textwidth)/2}
\setlength{\oddsidemargin}{(\mymargin)-1in}
\setlength{\evensidemargin}{(\mymargin)-1in}
% 设置正文区偏移量,奇数页向右偏,偶数页向左偏
\newlength \myshift
\setlength{\myshift}{0.35cm}	% 双面打印的奇偶页偏移值,可根据需要修改,建议小于 0.5cm
\addtolength{\oddsidemargin}{\myshift}
\addtolength{\evensidemargin}{-\myshift}
% 页眉页脚相关距离设置
\setlength{\topmargin}{-0.05cm}
\setlength{\headheight}{0.50cm}
\setlength{\headsep}{0.90cm}
\setlength{\footskip}{1.47cm}
% 公式的精调
\allowdisplaybreaks[4]  % 可以让公式在排不下的时候分页排,这可避免页面有大段空白。

%下面这组命令使浮动对象的缺省值稍微宽松一点,从而防止幅度
%对象占据过多的文本页面,也可以防止在很大空白的浮动页上放置很小的图形。
\renewcommand{\topfraction}{0.9999999}
\renewcommand{\textfraction}{0.0000001}
\renewcommand{\floatpagefraction}{0.9999}

%%%%%%%%%%%%%%%%%%%%%%%%%%%%%%%%%%%%%%%%%%%%%%%%%%%%%%%%%%%%%%%%%%%%%%
% 字体字号定义
%%%%%%%%%%%%%%%%%%%%%%%%%%%%%%%%%%%%%%%%%%%%%%%%%%%%%%%%%%%%%%%%%%%%%%
% 字号
\newcommand{\yihao}{\fontsize{26pt}{39pt}\selectfont}	  % 一号,1.5  倍行距
\newcommand{\xiaoyi}{\fontsize{24pt}{30pt}\selectfont}  % 小一,1.25 倍行距
\newcommand{\erhao}{\fontsize{22pt}{27.5pt}\selectfont} % 二号,1.25 倍行距
\newcommand{\xiaoer}{\fontsize{18pt}{22.5pt}\selectfont}% 小二,1.25 倍行距
\newcommand{\sanhao}{\fontsize{16pt}{20pt}\selectfont}  % 三号,1.25 倍行距
\newcommand{\xiaosan}{\fontsize{15pt}{19pt}\selectfont} % 小三,1.25 倍行距
\newcommand{\sihao}{\fontsize{14pt}{17.5pt}\selectfont} % 四号,1.25倍行距
\newcommand{\xiaosi}{\fontsize{12pt}{15pt}\selectfont}  % 小四,1.25倍行距
\newcommand{\dawu}{\fontsize{10.5pt}{18pt}\selectfont}  % 五号,1.75倍行距
\newcommand{\zhongwu}{\fontsize{10.5pt}{16pt}\selectfont}% 五号,1.5 倍行距
\newcommand{\wuhao}{\fontsize{10.5pt}{10.5pt}\selectfont}% 五号,单倍行距
\newcommand{\xiaowu}{\fontsize{9pt}{9pt}\selectfont}	   % 小五,单倍行距

\newcommand{\song}{\CJKfamily{song}}
\newcommand{\hei}{\CJKfamily{hei}}
\newcommand{\kai}{\CJKfamily{kai}}
\newcommand{\fs}{\CJKfamily{fs}}
\newcommand{\xkai}{\CJKfamily{xkai}}

% defaultfont 默认字体命令
\def\defaultfont{\renewcommand{\baselinestretch}{1.27}
\fontsize{12pt}{15pt}\selectfont}

% 设置目录字体和行间距
\def\defaultmenufont{\renewcommand{\baselinestretch}{1.22}
\fontsize{12pt}{15pt}\selectfont}




